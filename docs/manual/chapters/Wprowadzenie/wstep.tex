\section{Wstęp}
\paragraph{}Celem projektu jest wykonanie działającego modelu kontrolera lotu spełniającego wymgania (rozdział \ref{wymagania_sprzetowe}), wykonującego zadania niezbędne do prowadzenia skutecznego wypracowywania komend oraz procedur związanych ze wznoszeniem i utrzymaniem maszyn latających w powietrzu. Projekt ten będzie złożony z 3 podstawowych modułów: teoria lotu i sterowania, rozwiązanie elektroniczne (z ang. \textit{hardware}) oraz rozwiązanie programowe (z ang. \textit{software}). Wykorzystane komponenty oraz podzespoły zostaną dostatecznie opisane tak aby niniejszy dokument był instotnym źródłem niezbędnej wiedzy oraz instrukcji (samouczka) w próbach odtworzenia procesu lub skonstruowania podobnej konstrukcji na bazie własnych wymagań. Jednocześnie uprasza się o ostrożność w podejmowaniu wszelkich decyzji oraz działań w procesie twórczym oraz nie traktowania tego skrypu jako wyroczni (autor nie ponosi odpowiedzialności za odniesione szkody na zdrowiu oraz straty materialne). Wskazówki określone w tej pracy wynikają z doświadczeń prowadzenia podobnego projektu na przestrzeni lat. Wszelkie kroki podjęte w dokumencie są koniczne w celu stworzenia oraz kontynuowania pojektu na tak szeroką skalę z jednoczesnym rutunowym uczeniem się na bieżąco wszelkich technik lub umiejętności koniecznych (a nie posiadanych na etapie pracy). Dodatkowo wszystkie niezbędne materiały będą umieszczone we spisach na końcu pracy. Autor pracy zachęca do wnikiliwego podjęcia wyzwania i życzy sukcesów na polu realizowania inżynieryjnych projektów.
