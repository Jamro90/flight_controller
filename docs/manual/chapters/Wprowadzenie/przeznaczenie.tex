\section{Przeznaczenie}
\paragraph{}Głównym zadaniem projektu jest wykonananie działającego modelu kontrolera lotu, którego przeznaczeniem będzie obsługa podespołów pokoładowych bezzałogowych statków powietrznych. W tym celu należy przypomnieć wcześniej wykonany projekt o nazwie POLAR (Polowy Obserwacyjno Latający Aparat Rozpoznawczy) zaprezentowany na konkursie III edycji konkursu bezzałogowych statków powietrznych Ministerstwa Obrony Narodowej. Projekt ten polegał na fuzji pocisku moździeżowego z klasycznym układem quadrokoptera. Zadanie takiej platformy polegało na wykonaniu rozpoznania przy użyciu moździerzy LM60 w ramach szybkiego i skrytego rozpoznania i zasilenia systemu dowodzenia informacjami z pierwszej ręki prosto z pola walki. Konstrukcja skorupy zmuszona była do wytrzymania ciśnienia panującego w lufie moździerza oraz przeciążenia wynikające z nagłego skoku przyspieszeń w wyniku wybuchu ładunku miotającego. Poza trwałością części mechanicznych należało pamiętać o trwałości komponentów elektronicznych oraz o ich sztywności umieszcenia (potencjalnych dyslokacji czy przegrzewania). Rozmieszczenie wszystkich niezbędnych elemntów na niewielkiej przestrzeni (pojemność pocisku) oraz stabilność prowadzenia lotu (obserwacji). Jedną z najważnieszych decyzji było wybranie platformy odpowiadającej za prowadzenie obliczeń i wypracowywania sygnałów do sterowania i kontroli jednostki latającej.
